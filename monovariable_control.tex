\section{Lab 1}
\subsection{Motivation}\label{sec:lab1_mot}
The motivation behind the first lab task was to create a PD regulator to the linearized
model of the helicopter \ref{eq:lin_model} for a cause to control its pitch. The equations
of motion can be find in \ref{eq:model_1}. For a simple visualization of the helicopter, see \ref{fig:helicopter}.

\begin{subequations}\label{eq:model_1}
	\begin{align}
		\ddot{p}  &= -\frac{K_{f}l_{p}V_{d}}{J_{p}} = \frac{L_1Vd}{J_p} \label{eq:model_1_p} \\
		\ddot{e}  &= (2m_pgl_h-m_cgl_c)cos(e)+K_fl_hV_scos(p) = L_2cos(e)+L_3V_scos(p)\label{eq:model_1_e} \\
		\ddot{\lambda} &= \frac{l_hk_fV_scos(e)sin(p)}{J_{\lambda}} = L_4V_scos(e)sin(p) \label{eq:model_1_l} \\
	\end{align}
\end{subequations}

\subsection{Lab preperation}\label{sec:lab1_prep}
There were several tasks which needed to be done as preperation to the first lab day.
The first task was to derive the equations of motion from a given diagram of the helicopter \ref{pic:heli_model}.
This task required thoroughly analyzing the forces which the helicopter is affected by.
When this was done, it was needed to linearize the model around the equlibrium (all system states equal zero) represented by \ref{pic:heli_model} such that 
creating controllers would be easier. We were now ready to implement a PD controller given by \ref{eq:PD}
and insert this into the linearized equation of pitch acceleration, to control the pitch to a given reference.
The last preperation for lab 1 was to make a test plan for pole placements to the pitch control. We used the equations in \ref{eq:pole_place}.

\begin{subequations}\label{eq:lin_model}
	\begin{align}
        \tilde{V_d} &= V_d-V_{d,origo} = V_d\\
        \tilde{V_s} &= V_s-V_{s,origo} = V_s+\frac{L_2}{L_3}\\
		\ddot{p}  &= \frac{L_1\tilde{V_d}}{J_p} = K_1V_d\label{eq:lin_p} \\
		\ddot{e}  &= \frac{L_3\tilde{V_s}}{J_e} = K_2\tilde{V_s}\label{eq:lin_e} \\
		\ddot{\lambda} &= \frac{L_4L_2\tilde{p}}{L_3J_{\lambda}} = K_3p\label{eq:lin_l} \\
	\end{align}
\end{subequations}
\begin{subequations}\label{eq:PD}
	\begin{align}
        V_d &= K_{pp}(pc-p)-K_{pd}\dot{p}
	\end{align}
\end{subequations}
\begin{subequations}\label{eq:pole_place}
	\begin{align}
    K_{pd} &= -\frac{s_2+s_1}{K_1}\\
    K_{pp} &= \frac{s_2(s_2-s_1)}{K_1(\frac{s_2}{s_1}-1)}
	\end{align}
\end{subequations}

\subsection{Hypotheses/Test plan}
We wanted to expirement the controller by setting the poles at several different positions, including real and complex numbers.
We prepared twelve different tests, with hyoptheses coming from control theory which as for poles in LHS of s plane are exponentially stable, RHS are unstable and poles in origo are stable.
The tabular \ref{tab:testskjema_lab1} gives an overview of our tests and their s-values, stability result and hypotheses.

\begin{center}
    \begin{tabular}{||c c c c c||} 
     \hline
     Test & Poles & Pitch ref & Hypotheses & Result \\ [0.5ex] 
     \hline\hline
     1 & -1,-1 & 87837 & 787 &  \\ 
     \hline
     2 & -5,-5 & 78 & 5415 &  \\
     \hline
     3 & -1  & 778 & 7507 &  \\
     \hline
     4 & 545 & 18744 & 7560 &  \\
     \hline
     5 & 88 & 788 & 6344 &  \\ [1ex] 
     \hline
    \end{tabular}
    \end{center}
\subsection{Results}

\subsection{}

