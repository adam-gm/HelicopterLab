\documentclass[11pt, a4paper, USenglish]{article} % change ``USenglish'' to ``norsk'' if applicable.
\usepackage{subfig}
\usepackage{kyblab} % Contains all included packages. See kyblab.sty.
\usepackage{float}
\addbibresource{bibliography.bib} % Makes the bibliography file available to biblatex.

\begin{document}

% Titlepage
\title{Helicopter Lab Report }
\author{Group 53\\Adam Godtland Moumen\\Tobias De La Plaza Domaas\\Carl Joakim Duraj}
\date{20.11.2024}
\begin{titlepage}
    \maketitle
    \begin{figure}
    \centering
    \includegraphics[width=0.5\textwidth]{figures/itk_ntnu}\\
    Department of Engineering Cybernetics
    \end{figure}
    \thispagestyle{empty}
\end{titlepage}

% Abstract
\newpage
\input{abstract}    
\thispagestyle{empty} % Avoid page numbering on the abstract page.

% TOC
\newpage
\tableofcontents
\thispagestyle{empty} % Avoid page numbering on the table of contents.

% Main content
\newpage
\setcounter{page}{1}
\include{intro}
\section{Lab 1}
\subsection{Motivation}\label{sec:lab1_mot}
The motivation behind the first lab task was to create a PD regulator to the linearized
model of the helicopter \ref{eq:lin_model} for a cause to control its pitch. The equations
of motion can be find in \ref{eq:model_1}. For a simple visualization of the helicopter, see \ref{fig:helicopter}.

\begin{subequations}\label{eq:model_1}
	\begin{align}
		\ddot{p}  &= -\frac{K_{f}l_{p}V_{d}}{J_{p}} = \frac{L_1Vd}{J_p} \label{eq:model_1_p} \\
		\ddot{e}  &= (2m_pgl_h-m_cgl_c)cos(e)+K_fl_hV_scos(p) = L_2cos(e)+L_3V_scos(p)\label{eq:model_1_e} \\
		\ddot{\lambda} &= \frac{l_hk_fV_scos(e)sin(p)}{J_{\lambda}} = L_4V_scos(e)sin(p) \label{eq:model_1_l} \\
	\end{align}
\end{subequations}

\subsection{Lab preperation}\label{sec:lab1_prep}
There were several tasks which needed to be done as preperation to the first lab day.
The first task was to derive the equations of motion from a given diagram of the helicopter \ref{pic:heli_model}.
This task required thoroughly analyzing the forces which the helicopter is affected by.
When this was done, it was needed to linearize the model around the equlibrium (all system states equal zero) represented by \ref{pic:heli_model} such that 
creating controllers would be easier. We were now ready to implement a PD controller given by \ref{eq:PD}
and insert this into the linearized equation of pitch acceleration, to control the pitch to a given reference.
The last preperation for lab 1 was to make a test plan for pole placements to the pitch control. We used the equations in \ref{eq:pole_place}.

\begin{subequations}\label{eq:lin_model}
	\begin{align}
        \tilde{V_d} &= V_d-V_{d,origo} = V_d\\
        \tilde{V_s} &= V_s-V_{s,origo} = V_s+\frac{L_2}{L_3}\\
		\ddot{p}  &= \frac{L_1\tilde{V_d}}{J_p} = K_1V_d\label{eq:lin_p} \\
		\ddot{e}  &= \frac{L_3\tilde{V_s}}{J_e} = K_2\tilde{V_s}\label{eq:lin_e} \\
		\ddot{\lambda} &= \frac{L_4L_2\tilde{p}}{L_3J_{\lambda}} = K_3p\label{eq:lin_l} \\
	\end{align}
\end{subequations}
\begin{subequations}\label{eq:PD}
	\begin{align}
        V_d &= K_{pp}(pc-p)-K_{pd}\dot{p}
	\end{align}
\end{subequations}
\begin{subequations}\label{eq:pole_place}
	\begin{align}
    K_{pd} &= -\frac{s_2+s_1}{K_1}\\
    K_{pp} &= \frac{s_2(s_2-s_1)}{K_1(\frac{s_2}{s_1}-1)}
	\end{align}
\end{subequations}

\subsection{Hypotheses/Test plan}
We wanted to expirement the controller by setting the poles at several different positions, including real and complex numbers.
We prepared twelve different tests, with hyoptheses coming from control theory which as for poles in LHS of s plane are exponentially stable, RHS are unstable and poles in origo are stable.
The tabular \ref{tab:testskjema_lab1} gives an overview of our tests and their s-values, stability result and hypotheses.

\begin{center}
    \begin{tabular}{||c c c c c||} 
     \hline
     Test & Poles & Pitch ref & Hypotheses & Result \\ [0.5ex] 
     \hline\hline
     1 & -1,-1 & 87837 & 787 &  \\ 
     \hline
     2 & -5,-5 & 78 & 5415 &  \\
     \hline
     3 & -1  & 778 & 7507 &  \\
     \hline
     4 & 545 & 18744 & 7560 &  \\
     \hline
     5 & 88 & 788 & 6344 &  \\ [1ex] 
     \hline
    \end{tabular}
    \end{center}
\subsection{Results}

\subsection{}


\section{Lab 2 - Multivariable control}
\subsection{Introduction}
As the helicopter is a complex system with multiple states, it is reasonable to implement a multivariable control system.
The system is controlled by an LQR-controller, as it provides an intuitive way to design a control rule based on minimizing a cost function. 
Analyzing the controllability matrix of the system showed that the system was indeed controllable.
Because the LQR controller provides the feedback gain maktrix $K$ which minimizes a cost function based on the states and input, it will naturally try to drive the states towards 0. 
In order to actually have the states reach their target reference, a feed forward input is also implemented. 
Lastly because our system, and also thus our control rules, are based on a linearized approximation of the real system there has been added an integral action to remove the steady-state error.

The system matrices (before the integral action) are shown in \ref{eq:lab2-system1}.
\begin{equation}
\label{eq:lab2-system1}
   A = \begin{bmatrix}
	0 & 1 & 0 \\
	0 & 0 & 0 \\
	0 & 0 & 0
   \end{bmatrix},\; 
   B = \begin{bmatrix}
	0 & 0 \\
	0 & K_1 \\
	K_2 & 0
   \end{bmatrix}
\end{equation}
Using this we can calculate the controllability matrix \ref{eq:lab2-controllability}, which has a rank of 3 and the system is thus controllable. 
\begin{equation}
\label{eq:lab2-controllability}
   \mathcal{C} = \begin{bmatrix}
	0 & 0 & K_1 & 0 & 0 \\
	0 & K_1 & 0 & 0 & 0 \\
	K_2 & 0 & 0 & 0 & 0
   \end{bmatrix}
\end{equation}
Our control rule is then given by \ref{eq:lab2-control-rule}, where $K$ is given by LQR and $F$ is derived in \ref{eq:lab2-F} from $K$ to drive the system to our target reference $r$.
\begin{equation}
\label{eq:lab2-control-rule}
   u = -Kx + Fr
\end{equation}
\begin{equation}
\label{eq:lab2-F}
   F = \begin{bmatrix}
	K_{11} & K_{13} \\
	K_{21} & K_{23}
   \end{bmatrix}
\end{equation}
The system matrices (after the integral action) are shown in \ref{eq:lab2-system2}. 
The G matrix here is being multiplied the the reference.
\begin{equation}
\label{eq:lab2-system2}
	A = \begin{bmatrix}
		0 & 1 & 0 & 0 & 0 \\
		0 & 0 & 0 & 0 & 0 \\
		0 & 0 & 0 & 0 & 0 \\
		-1 & 0 & 0 & 0 & 0 \\
		0 & 0 & -1 & 0 & 0 
	\end{bmatrix},\;
	B = \begin{bmatrix}
		0 & 0 \\
		0 & K_1 \\
		K_2 & 0 \\ 
		0 & 0 \\ 
		0 & 0
	\end{bmatrix},\; 
	G = \begin{bmatrix}
		0 & 0 \\
		0 & 0 \\
		0 & 0 \\
		1 & 0 \\ 
		0 & 1
	\end{bmatrix}
\end{equation}

\subsection{Hypothesis and test plan}
All our test were conducted the same way with a test plan structured as this: 
\begin{itemize}
	\item Start up the system.
	\item Apply a change in the reference from 0 to $\frac{\pi}{4}$ between time 3s and 5s. 
	\item Simulate a disturbance by applying an external pulse in the input voltage at 13s.
\end{itemize}
The main things we wanted to test was how different choices of Q and R affected the sytem as well as how the integral action affected the system. 
We came up with 5 cases to test the relation between Q and R, and tested these 5 cases both with and without integral action:
\begin{itemize}
	\item $R = 5\mathbf{I},\; Q=\mathbf{I}$
	\item $R = 100\mathbf{I},\; Q=\mathbf{I}$
	\item $R = \mathbf{I},\; Q=5\mathbf{I}$
	\item $R = \mathbf{I},\; Q=100\mathbf{I}$
	\item $R = \mathbf{I},\; Q=\mathbf{I}$
\end{itemize}
Our Hypothesis were that an R larger than Q would give a slow system response as the cost of the input was weighted higher than the cost of the state, meanwhile the opposite would give a fast response. 
When both Q and R were equal to an identity matrix we hypothesized that we would get some form of balanced response. 
As for with or without integral gain our hypothesis was that the integral gain would remove the steady state error of the system.

\subsection{Results}
Here are the test results for all the cases

\subsubsection*{$R = 5\mathbf{I},\; Q=\mathbf{I}$}
The pitch response fails to even respond to the reference at all without the integral action. The integral helps some but the response is still very sluggish.
\begin{figure}[H]
	\centering
	\includegraphics[width=0.6\textwidth]{figures/lab2-test1.png}
	\caption{Test result for test 1}
	\label{}
\end{figure}

\subsubsection*{$R = 100\mathbf{I},\; Q=\mathbf{I}$}
As with the previous test, the response without integral fails to respond at all. THe integral helps some, but is way too slow to even reach the reference point.
\begin{figure}[H]
	\centering
	\includegraphics[width=0.6\textwidth]{figures/lab2-test2.png}
	\caption{Test result for test 2}
	\label{}
\end{figure}

\subsubsection*{$R = \mathbf{I},\; 5Q=\mathbf{I}$}
Here the pitch is able to track the reference better, but does converge to a stationary error without the integral. The integral does overshoot a bit, but converges after to the reference. This can also be seen towards the end of the series. 
\begin{figure}[H]
	\centering
	\includegraphics[width=0.6\textwidth]{figures/lab2-test3.png}
	\caption{Test result for test 3}
	\label{}
\end{figure}

\subsubsection*{$R = \mathbf{I},\; 100Q=\mathbf{I}$}
The reference is being followed by both the controllers. The integral action eliminates the stationary error.
\begin{figure}[H]
	\centering
	\includegraphics[width=0.6\textwidth]{figures/lab2-test4.png}
	\caption{Test result for test 4}
	\label{}
\end{figure}

\subsubsection*{$R = \mathbf{I},\; Q=\mathbf{I}$}
Both signals track the reference, but both controllers are a bit slower to reach it. Also the controller with no integral action has a stationary error.
\begin{figure}[H]
	\centering
	\includegraphics[width=0.6\textwidth]{figures/lab2-test5.png}
	\caption{Test result for test 5}
	\label{}
\end{figure}

\subsection{Conclusion}
All our hypotheses matched with our findings. The tests with high R and low Q had weak, in fact too weak, responses. The tests with high Q values and low R had very fast responses, while the test with equal Q and R values had converging but somewhat slow responses. In addition it was shown that the integral action helped remove the staionary error.
\section{Lab 4 - Kalman filter}
\subsection{Motivation}
In this lab we are supposed to dive in to the realm of an other state estimator, Kalman filter. It is tuned based on process noise and measurment noise, and will be used to a stochastic model of the system.
Unlike the Luenberger observer which calculates a constant gain, the kalman gain is computed and varies with each timestep. This lab takes us through discretization 
of the state space system and expirementation of measurement covariance and covariance of disturbance. 

\subsection{Lab preperation}
Our first task in the lab preperation was to derive the discrete system model of the state space model in \ref{eq:state_space_model}.
The discrete system model is at the form \ref{eq:discrete_model} where as the matrix equations can be found in \ref{eq:discrete_matrices}.

\begin{subequations}\label{eq:discrete_model}
    \begin{align}
    x[k+1] &= A_d x[k] + B_d u[k] + \omega_d[k] \label{eq:discrete_x} \\
    y[k] &= C_d x[k] + v_d[k] \label{eq:discrete_y} \\
    \omega_d &\sim \mathcal{N}(0, Q_d), \quad v_d \sim \mathcal{N}(0, R_d) \label{eq:discrete_noise}
    \end{align}
\end{subequations}


\begin{subequations}\label{eq:discrete_matrices}
    \begin{align}
    A_d &= e^{AT} = \begin{bmatrix}
    1 & 1 & 0 & 0 & 0 & 0\\
    0 & 1 & 0 & 0 & 0 & 0\\
    0 & 0 & 1 & 1 & 0 & 0\\
    0 & 0 & 0 & 1 & 0 & 0\\
    \frac{K3}{2} & \frac{K3}{6} & 0 & 0 & 1 & 1\\
    K3 & \frac{K3}{2} & 0 & 0 & 0 & 1\\
    \end{bmatrix} \label{eq:discrete_Ad}\\
    B_d &= \int\limits_0^T e^{A\alpha} \, d\alpha \, B \ = \begin{bmatrix}
    0 & \frac{K1T^2}{2} \\
    0 & K1T \\
    \frac{K2T^2}{2} & 0 \\
    K2T & 0 \\
    0 & \frac{K1K3T^4}{24} \\
    0 & \frac{K1K3T^3}{6} \\
    \end{bmatrix} \label{eq:discrete_Bd}\\
    C_d &= C = \begin{bmatrix}
        1 & 0 & 0 & 0 & 0 & 0\\
        0 & 1 & 0 & 0 & 0 & 0\\
        0 & 0 & 1 & 0 & 0 & 0\\
        0 & 0 & 0 & 1 & 0 & 0\\
        0 & 0 & 0 & 0 & 0 & 1\\
    \end{bmatrix}\label{eq:discrete_Cd}
    \end{align}
\end{subequations}


\subsection{Hypotheses/Test plan}
For the tests we were supposed to change the Qd matrix, which has a role of specifying how much the Kalman filter should trust the model. A small Qd 
means high trust in the model, while a high Qd means low trust and faster reaction to measurement updates. 
\begin{table}[h]
    \centering
    \phantomsection % Creates an anchor for the hyperlink
     % Place the label at the top of the table
        \begin{tabular}{||c c c c c||} 
         \hline
         Test & Poles & Pitch ref & Hypotheses & Result \\ [0.5ex] 
         \hline\hline
         1 & -1,-1 & 0 & Exp. Stability & Exp. Stability \\ 
         \hline
         2 & -5,-5 & 0 & Exp. Stability & Exp. Stability \\
         \hline
         3 & $-1\pm j5$  & 0 & Exp. Stability & Exp. Stability  \\ [1ex]
         \hline
        \end{tabular}
        \label{tab:testskjema_lab4}
        \caption{Test scheme lab4}
    \end{table}
\subsection{Results}
Because of the IMU (which is used to give measurements to the Kalman filter) has a lower 
sample rate than Simulink, we were needed to adjust the Kalman filter program to that. 
When there were duplicated measurements as input to the Kalman filter, we made the corrected
state estimate and error covariance to be equal to the predicted state estimate and error covariance. See \ref{eq:slow_sample}.

\begin{subequations}\label{eq:slow_sample}
    \begin{align}
    \hat{x} &= \bar{x}  \label{eq:bad_sample_x} \\
    \hat{P} &= \bar{P} \label{eq:bad_sample_p} \\
    \end{align}
\end{subequations}

The first task at the lab was to find an estimate on measurement noise. To do this we 
created two different time series, one for the helicopter laying still and one for the helicopter in the linearization point.
The estimated measurement noise used for the task corresponds to when the helicopter is stationary in its linearization point.
The resulted covariance matrices can be seen in \ref{eq:noise_est}. Rg is for helicopter on ground and Rd when helicopter in stationary linearization point.
Taking a glance at the two matrices they seem to be nearly identical, with only a few decimal difference in som covariances. 
From this it seems that it is indifferent which of them is chosen for the Kalman filter.
\begin{subequations}\label{eq:noise_est}
    \begin{align}
    R_f &=  \begin{bmatrix}
    2.9398 & -0.0135 & -0.3548 & 0.0031 & -0.0667\\
    -0.0135 &   0.0001  &  0.0016  & -0.0000  &  0.0003\\
   -0.3548  &  0.0016  &  0.0428  & -0.0004  &  0.0080\\
    0.0031  & -0.0000 &  -0.0004 &   0.0000  & -0.0001\\
   -0.0667  &  0.0003  &  0.0080 &  -0.0001  &  0.0015\\
    \end{bmatrix}\label{eq:Rg}\\
    R_d &= \begin{bmatrix}
        2.9546  & -0.0147  & -0.3545  &  0.0024   &-0.0668\\
        -0.0147 &   0.0001 &   0.0018 &  -0.0000  &  0.0003\\
        -0.3545 &   0.0018  &  0.0425 &  -0.0003  &  0.0080\\
         0.0024 &  -0.0000  & -0.0003 &   0.0000  & -0.0001\\
        -0.0668 &   0.0003  &  0.0080 &  -0.0001  &  0.0015\\
    \end{bmatrix}\label{eq:Rd}
    \end{align}
\end{subequations}



\subsection{Conclusion}


\section{Part III - Luenberger observer}



The Luenberger observer is introduced to improve the state estimation process in the helicopter lab. In previous labs, we relied solely on the Inertial Measurement Unit(IMU) for measurements like angular rates and accelerations. While useful, the IMU suffers from sensor noise, errors in angular estimation, and sensitivity to its position near the lever arm, which affects data accuracy.
\vspace{1em}

Positioning of the IMU influences the data. And the IMU is centered more in the middle of the lever arm, which influences the data.
By use of the Luenberger observer mathematical model we can improve the measurements.
\vspace{1em}

\section{Lab Preparation}
\title{\textbf{Extended state-space formulation}}
\maketitle
\vspace{1em}


Given the basis of
\[
x =
\begin{bmatrix}
{p} \\
\dot{p}\\
{e} \\
\dot{e} \\

\dot{{\lambda}} \\


\end{bmatrix}, \quad
u =
\begin{bmatrix}
\tilde{V}_s \\
V_d
\end{bmatrix}, \quad
\]

Based on the variables we get the following state-space model matrixes:

\[
A = \begin{bmatrix}
0 & 1 & 0 & 0 & 0 \\
0 & 0 & 0 & 0 & 0 \\
0 & 0 & 0 & 1 & 0 \\
0 & 0 & 0 & 0 & 0 \\
K_3 & 0 & 0 & 0 & 0 \\
\end{bmatrix}, \quad
B = \begin{bmatrix}
0 & 0 \\
0 & K_1 \\
0 & 0 \\
K_2 & 0 \\
0 & 0 \\
\end{bmatrix}, \quad
C = \begin{bmatrix}
0 & 0 & 1 & 0 & 0 \\
0 & 0 & 0 & 0 & 1 \\

\end{bmatrix}.
\]

\vspace{2em}
\noindent
\noindent
\title{\textbf{Observability}}

\maketitle
\vspace{2em}

The observability matrix is the following:
\[
O =
\begin{bmatrix}
C \\
CA \\
\vdots \\
CA^{n-1}
\end{bmatrix}
\]


\noindent
The system is Observable. Which makes it possible to infer states of the system. So you can effectively monitor or control the system.
The minimum set of states you need is e, and lambda that still makes the system observable. 


\vspace{1em}


\noindent
\title{ \textbf{State estimator}} 
\maketitle
\vspace{1em}
\vspace{1em}

\[
\dot{\hat{x}} = \hat{A}x + Bu + L(y - \hat{C}x)
\]

\vspace{1em}



\noindent
We just used the 
L=place(A',C',P) command in matlab. P is a vector of the wanted poles.
Like the following example:

\[
P = \begin{bmatrix}
-9 \\
-9 \\
-9 \\
-9 \\
-9
\end{bmatrix}
\]

\vspace{1em}



\noindent
 Since the system is observable, it is possible to place the poles of the estimator arbitrarily
 by choosing an appropriate gain matrix, L. The poles of the estimator is the same as
 the poles of A-LC. 
 

 \vspace{1em}
\noindent
 \title{{\textbf{Test plan}}
 \maketitle
 \vspace{1em}


Trying different values of poles. 
From small values to bigger in the Left Half Plane.
We tried pole values: -2, -9 ,-90.
The reasoning was that smaller values of poles will correspond to slower dynamics and a slower convergence rate. On the other hand model uncertainties will be increased with too big values of poles. The helicopter lab has states that change quickly. So we thought bigger pole values would be better.
The first value was bad, the second was okay, and the third had a good smooth result but had a offset.
\vspace{2em}
For the smaller pole values the estimates are less accurate. And for the larger pole values there is a offset(lag). 
For some estimates the difference is hard to spot like for the Apples.
The difference between small and big pole values can clearly be seen between Blueberry 1 and Blueberry 4. 

\vspace{2em}

\title{\textbf{Conclusion}}
\maketitle
\vspace{2em}






\noindent
The pitch estimations appears to be more inaccurate the faster you try to
to sample a state, and limitations to the observer has to be implemented to
make the system stable.
The L matrix in the observer scales the deviation between the measured and
estimated states.
\noindent
Based on the plots, the elevation estimation seems to be working correctly as
it has very few deviations from the actual measurement.


\includegraphics{LabTest1(Apple)fixed.png}
\includegraphics{LabTest2(Apple)fixed.png}

\includegraphics{LabTest3(Apple)fixed.png}
\includegraphics{LabTest4(Apple)fixed.png}


\includegraphics{LabTest1(Blueberry).png}
\includegraphics{LabTest4(Blueberry).png}


\include{conclusion}
\addcontentsline{toc}{section}{Appendix} % Remove this if you don't want the appendix included in the table of contents.
\appendix

\section{Helicopter diagram}
\begin{figure}[htb]
	\centering
		\includegraphics[width = \textwidth]{figures/forces.pdf}
	\caption{Helicopter diagram.}
\label{fig:helicopter_dia}
\end{figure}



% \input simply inserts the contents of the file, while \include forces a \newpage.
% See \input vs. \include: http://tex.stackexchange.com/questions/246/when-should-i-use-input-vs-include

% References
\newpage
\addcontentsline{toc}{section}{References}
\printbibliography{}
\label{sec:bibliography}

\end{document}
